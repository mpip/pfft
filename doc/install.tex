%%%%%%%%%%%%%%%%%%%%%%%%%%%%%%%%%%%%%%%%%%%%%%%%%%%%%%%%%%%%%%%%%%%%%%%%%%%%%%%
\chapter{Installation}\label{chap:inst}
%%%%%%%%%%%%%%%%%%%%%%%%%%%%%%%%%%%%%%%%%%%%%%%%%%%%%%%%%%%%%%%%%%%%%%%%%%%%%%%




In the simplest case, your hardware platform will be recognized by the PFFT configure script automatically,
so all we have to do is
\begin{lstlisting}[escapechar=§]
wget http://www.tu-chemnitz.de/~mpip/software/pfft-§\pfftversionsl§.tar.gz
tar xzvf pfft-§\pfftversion§.tar.gz
cd pfft-§\pfftversion§
./configure
make
\end{lstlisting}

If the \fftw\fftwversion{} software library is already installed on your system but not found by the configure script,
you can provide the FFTW installation directory \code{\$FFTWHOME} to configure by
\begin{lstlisting}
./configure --with-fftw3=$FFTWHOME
\end{lstlisting}
or specify the FFTW include directory \code{\$FFTWINC} and FFTW library directory \code{\$FFTWLIB} separately via
\begin{lstlisting}
./configure --with-fftw3-includedir=$FFTWINC --with-fftw3-libdir=$FFTWLIB
\end{lstlisting}
To install PFFT in a user specified directory \code{\$PFFTINSTDIR} call configure with the option
\begin{lstlisting}
./configure --prefix=$PFFTINSTDIR
\end{lstlisting}


\section{Install the Latest Official Release of FFTW}

\begin{lstlisting}[escapechar=§]
wget http://www.fftw.org/fftw-§\fftwversionsl§.tar.gz
tar xzvf fftw-§\fftwversion§.tar.gz
cd fftw-§\fftwversion§
./configure --enable-mpi --prefix=$HOME/local/fftw3_mpi
make
make install
\end{lstlisting}

\begin{lstlisting}[escapechar=§]
wget http://www.tu-chemnitz.de/~mpip/software/pfft-§\pfftversionsl§.tar.gz
tar xzvf pfft-§\pfftversion§.tar.gz
cd pfft-§\pfftversion§
./configure --with-fftw3=$HOME/local/fftw3_mpi
make
\end{lstlisting}




\begin{lstlisting}
../configure --with-fftw3=$FFTWDIR
\end{lstlisting}
where \code{FFTWDIR} stands for the install directory of FFTW. This is eqivalent to
\begin{lstlisting}
../configure --with-fftw3-libdir=$FFTWDIR/lib --with-fftw3-includedir=$FFTWDIR/include
\end{lstlisting}
but the second call gives the flexibility to declare arbitrary include and lib directories.


\begin{compactitem}
  \item[\mybox] describe complete install procedure with shell (like I did for Heraeus summer school and Joseba Alberdi)
  \item[\mybox] explain usage of configure script
  \item[\mybox] make shure that the preprocessor can find \code{fftw3.h} and \code{fftw3-mpi.h}
  \item[\mybox] \verb+#include <complex.h> #include <pfft.h>+
  \item[\mybox] use a MPI compiler wrapper, or make mpi libraries and header available
\end{compactitem}
