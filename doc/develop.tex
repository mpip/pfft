%%%%%%%%%%%%%%%%%%%%%%%%%%%%%%%%%%%%%%%%%%%%%%%%%%%%%%%%%%%%%%%%%%%%%%%%%%%%%%%
\chapter{Developers Guide}\label{chap:develop}
%%%%%%%%%%%%%%%%%%%%%%%%%%%%%%%%%%%%%%%%%%%%%%%%%%%%%%%%%%%%%%%%%%%%%%%%%%%%%%%



\section{Search and replace patterns}

Correct alignment of pfft.h header
\begin{lstlisting}
%s/^\(    [^ ]\+[^\\]*\)  \\/  \1\\/g  
\end{lstlisting}

Expand most macros of pfft.h to generate the function reference of this manual:
\begin{lstlisting}[language=bash,prebreak=\textbackslash,]
sed -e 's/ *\\$//g' -e 's/PFFT_EXTERN //g' \
    -e 's/PX(\([^)]*\))/pfft_\1/g' -e 's/ INT/ ptrdiff_t/g' \
    -e 's/ R/ double/g' -e 's/ C/ pfft_complex/g' \
    -e 's/^  //g' pfft.h > pfft.h.expanded
\end{lstlisting}








%%%%%%%%%%%%%%%%%%%%%%%%%%%%%%%%%%%%%%%%%%%%%%%%%%%%%%%%%%%%%%%%%%%%%%%%%%%%%%%
\chapter{ToDo}\label{chap:todo}
%%%%%%%%%%%%%%%%%%%%%%%%%%%%%%%%%%%%%%%%%%%%%%%%%%%%%%%%%%%%%%%%%%%%%%%%%%%%%%%

\begin{itemize}
  \item \code{PFFT_FORWARD} is defined as \code{FFTW_FORWARD}
  \item \code{FFTW_FORWARD} is defined as $-1$
  \item PFFT allows to chose between \code{FFTW_FORWARD} and \code{FFTW_BACKWARD}, which is not implemented by FFTW.
  \item Matlab uses the same sign convention, i.e., $-1$ for \code{fft} and $+1$ for \code{ifftn}
\end{itemize}

\section{Pruned FFT and Shifted Index Sets}
\subsection{Pruned FFT}
For pruned r2r- and c2c-FFT are defined as
\begin{equation*}
  g_l = \sum_{k=0}^{n_i-1} \hat g_k \eim{kl/n}, \quad l=0,\hdots,n_o-1,
\end{equation*}
where $n_i\le n$ and $n_o\le n$.

\subsection{Shifted Index Sets}
For $N\in 2\N$ we define the FFT with shifted inputs


For $K,L,N\in 2\N$, $L<N$, $L<N$ we define





\code{FFT_SHIFTED_IN} 



\section{Some Comments on MPI}
Following the MPI standard~\cite{MPI-2.2} we use the term process to denote the smallest MPI processing unit of a parallel, distributed memory machine.
The term process abstracts form the widely used terms processor, node and core.
Note that the term process abstracts the code development from the real physical architecture. For example we can \\

\emph{An MPI program consists of autonomous processes, executing their own code, in an MIMD style.}
( MPI: A Message-Passing Interface Standard, page 27)



