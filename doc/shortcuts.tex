\makeatletter
\providecommand{\IfPackageLoaded}[2]{\@ifpackageloaded{#1}{#2}{}}
\providecommand{\IfPackageNotLoaded}[2]{\@ifpackageloaded{#1}{}{#2}}
\providecommand{\IfElsePackageLoaded}[3]{\@ifpackageloaded{#1}{#2}{#3}}
\makeatother


% Shortcuts for math symbols.
\newcommand{\N}{{\mathbb{N}}}
\newcommand{\T}{{\mathbb{T}}}
\renewcommand{\S}{{\mathbb{S}}}
\newcommand{\NZ}{{\mathbb{N}_{0}}}
\newcommand{\Z}{{\mathbb{Z}}}
\newcommand{\R}{{\mathbb{R}}}
\newcommand{\Rp}{{\mathbb{R}_{+}}}
\newcommand{\Rn}{{\mathbb{R}^n}} 
\newcommand{\Rnn}{{\mathbb{R}^{n \times n}}}
\newcommand{\C}{{\mathbb{C}}}
\newcommand{\cO}{{\mathcal{O}}}
\newcommand{\tT}{{\text{T}}}
\newcommand{\ti}{{\text{i}}}
\newcommand{\e}{{{\mathrm{e}}}}
\newcommand{\eim}[1]{{\mathrm{e}^{-2\pi{\ti} #1}}}
\newcommand{\eip}[1]{{\mathrm{e}^{+2\pi{\ti} #1}}}
\renewcommand{\mathbf}[1]{{\boldsymbol{#1}}}
\newcommand{\ds}{\displaystyle}
\newcommand{\sinc}{\textrm{sinc}}
\newcommand{\dist}{\textrm{dist}}
\newcommand{\erf}{\textrm{erf}}
\newcommand{\erfc}{\textrm{erfc}}
\newcommand{\adj}{{\vdash \hspace*{-1.72mm} \dashv}}
\DeclareMathOperator{\diag}{diag}
\DeclareMathOperator{\dd}{d}
\newcommand{\bend}
{
  \hspace*{0ex} \hfill 
  \hbox{\vrule height 1.5ex \vbox{\hrule width 1.4ex \vskip
  1.4ex\hrule  width 1.4ex} \vrule height 1.5ex}
}
\DeclareMathOperator*{\argmin}{arg\,min}
\DeclareMathOperator*{\argmax}{arg\,max}
\newcommand{\BIGOP}[1]{\mathop{\mathchoice%
{\raise-0.22em\hbox{\huge $#1$}}%
{\raise-0.05em\hbox{\Large $#1$}}{\hbox{\large $#1$}}{#1}}}
\newcommand{\bigtimes}{\BIGOP{\times}}
\def\invisible#1{\textcolor{white}{#1}}
\newcommand{\Vect}[1]{{\mathbf{#1}}}
\newcommand{\Mat}[1]{{\mathbf{#1}}}
\newcommand{\Cal}[1]{{\mathcal{#1}}}
\newcommand{\fft}{\textsf{FFT}}

\newcommand{\ousetarrow}[2]{\overset{\textsf{#1}}{\underset{\textsf{#2}}{\rightarrow}}}
\newcommand{\osetarrow}[1]{\overset{\textsf{#1}}{\rightarrow}}
\newcommand{\usetarrow}[1]{\underset{\textsf{#1}}{\rightarrow}}
\newcommand{\fftarrow}[2]{\overset{\textsf{FFT#1}}{\underset{\textsf{#2}}{\rightarrow}}}
\newcommand{\transposearrow}[1]{\overset{\textsf{T}}{\underset{\textsf{#1}}{\rightarrow}}}

\newcommand{\nfft}{\textsf{NFFT}}
\newcommand{\nfftversion}{3.2.0}
\newcommand{\nfftversionsl}{\textsl{\nfftversion}}

\newcommand{\fftw}{\textsf{FFTW}}
\newcommand{\fftwversion}{3.3.3}
\newcommand{\fftwversionsf}{\textsf{\fftwversion}}
\newcommand{\fftwversionsl}{\textsl{\fftwversion}}

\newcommand{\pfft}{\textsf{PFFT}}
\newcommand{\pfftversion}{1.0.7-alpha}
\newcommand{\pfftversionsf}{\textsf{\pfftversion}}
\newcommand{\pfftversionsl}{\textsl{\pfftversion}}

\newcommand{\pnfft}{\textsf{PNFFT}}
\newcommand{\pnfftversion}{1.0.3-alpha}
\newcommand{\pnfftversionsl}{\textsl{\pnfftversion}}

\newcommand{\ppnfft}{\textsf{P$^2$NFFT}\xspace}
\newcommand{\ppnfftcap}{\texorpdfstring{\ppnfft}{\textsf{P\texttwosuperior NFFT}\xspace}}
\newcommand{\pppm}{\textsf{P$^3$M}\xspace}
\newcommand{\pppmcap}{\texorpdfstring{\pppm}{\textsf{P\textthreesuperior M}\xspace}}

\newcommand{\mybox}{$\Box$}
\newcommand{\myxbox}{{\Large\Crossedbox}}
\newcommand{\mycbox}{{\Large\Checkedbox}}
\newcommand{\myitem}{\item[\mybox]}
\newcommand{\myxitem}{\item[\myxbox]}
\newcommand{\mycitem}{\item[\mycbox]}

\newcommand{\mpipurl}{www.tu-chemnitz.de/~mpip}
\newcommand{\webmpip}{\url{\mpipurl}\xspace}
\newcommand{\websoft}{\url{\mpipurl/software.php}\xspace}
\newcommand{\webpfft}{\url{https://github.com/mpip/pfft.git}\xspace}
\newcommand{\webpnfft}{\url{https://github.com/mpip/pnfft.git}\xspace}


\newcommand{\red}[1]{\textcolor{red}{#1}}

\newcommand{\algname}{Alg.}
\newcommand{\eqname}{Eq.}
\newcommand{\figname}{Fig.}

% \theoremstyle{plain}
%% german
% \newtheorem{Satz}{Satz}[section] % define Satz enviroment for german docs
% \newtheorem{Def}[Satz]{Definition}
% \newtheorem{Lemma}[Satz]{Lemma}
% \newtheorem{Kor}[Satz]{Korollar}
% \newtheorem{Beh}[Satz]{Behauptung}
% \newtheorem{Bsp}[Satz]{Beispiel}
% \newtheorem{Bem}[Satz]{Bemerkung}
% \newenvironment{Beweis}
% {\begin{proof}[Beweis:]}
% {\begin{proof}[\bfseries\upshape Beweis:]}
% {\end{proof}}

%% english ???
%? \newtheorem{Satz}{Satz}[chapter]
%? \newtheorem{Def}[Satz]{Definition}
%? \newtheorem{Lemma}[Satz]{Lemma}
%? \newtheorem{Kor}[Satz]{Korollar}
%? \newtheorem{Beh}[Satz]{Behauptung} 
%? \newtheorem{Bsp}[Satz]{Beispiel} 
%? \newtheorem{Bem}[Satz]{Bemerkung} 
%? \newenvironment{Beweis}
%? % {\begin{proof}[Beweis:]}
%? {\begin{proof}[\bfseries\upshape Beweis:]}
%? {\end{proof}}

% \renewcommand{\qedsymbol}{\#}


%% German commands for algorithms
% \renewcommand{\algorithmicrequire}{\textbf{Eingabe:}}
% \renewcommand{\algorithmicensure}{\textbf{Ausgabe:}}

% activate for listings enviroment
\IfPackageLoaded{listings}
{
  \renewcommand{\ttdefault}{pcr}
  % \renewcommand{\lstlistingname}{\bfseries Quelltext}
  \lstset{language = C,
       morekeywords={ptrdiff_t, fftw_complex, pfft_plan, pfft_gcplan, pfft_timer, pfft_complex, MPI_Comm, size_t, INT, pfft_gctimer,
		     pfftf_complex, double, float, long double},
% 	morekeywords={ptrdiff_t, fftw_complex, pfft_plan, pfft_gcplan, pfft_timer, pfft_complex, MPI_Comm, size_t, INT,
% 	              PX, R, C, X, XM,
% 	              MPI_Init, MPI_Finalize, MPI_Comm_free,
% 	              pfft_init, pfft_destroy_plan, pfft_free,
% 	              pfft_execute,
% 	              pfft_init_input_c2c_3d,
% 	              pfft_plan_dft_3d,
% 	              pfft_alloc_complex,
% 	              pfft_create_procmesh_2d, pfft_create_procmesh,
% 	              pfft_local_size_dft_3d},
        deletekeywords={sizeof,return},
        basicstyle = \ttfamily\small,
	identifierstyle=,
	commentstyle = \slshape\footnotesize,
 	keywordstyle = \bfseries,
	backgroundcolor = \color{LightGray!40},
% 	numbers = left,
	breaklines = true,
  %   	xleftmargin = 3em,
  %   	prebreak = {...},
	breakatwhitespace = true,
	showstringspaces = false,
	escapeinside = {(*@}{@*)},
	moredelim=**[is][\color{red}]{(red@}{@*)},
        moredelim=**[is][\color{green}]{(gre@}{@*)},
        moredelim=**[is][\color{blue}]{(blu@}{@*)},
%         belowskip=0pt,
	}
  % 
  % \lstdefinestyle{txt}{language= C, 
  %       basicstyle = \sffamily\small,
  %       breaklines = true,
  %       backgroundcolor = \color{LightGray!0},
  %       numbers=none,
  %        xleftmargin = 0em,
  %       showstringspaces = false}
  % 
  % \lstdefinestyle{head}{
  %       backgroundcolor = \color{LightGray!0},
  %       numbers=none,
  %       belowskip= -2ex}
  % 
%   \newcommand{\code}[1]{\linebreak[2]{\ttfamily #1}}
   \newcommand{\code}[1]{:code:`#1`}
%  \newcommand{\code}[2][\empty]{\ifthenelse{\equal{#1}{\empty}}{\lstinline!#2!}{\lstinline[#1]!#2!}}
}

%% own enviroment for case differentiation
% \newenvironment{FUS}%
% { \renewcommand{\theenumi}{\arabic{enumi}}
%   \renewcommand{\labelenumi}{\itshape Fall \theenumi:}
%   \renewcommand{\theenumii}{\arabic{enumii}}
%   \renewcommand{\labelenumii}{\itshape Fall \theenumi.\theenumii:}
%   \renewcommand{\theenumiii}{\arabic{enumiii}}
%   \renewcommand{\labelenumiii}{\itshape Fall \theenumi.\theenumii.\theenumiii:}
%   \setlength{\labelsep}{1em}
%   \begin{compactenum}
% }
% {\end{compactenum}}
% \settowidth{\labelwidth}{\textit{Fall 44:}}
